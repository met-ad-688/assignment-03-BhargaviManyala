% Options for packages loaded elsewhere
\PassOptionsToPackage{unicode}{hyperref}
\PassOptionsToPackage{hyphens}{url}
\PassOptionsToPackage{dvipsnames,svgnames,x11names}{xcolor}
%
\documentclass[
  letterpaper,
  DIV=11,
  numbers=noendperiod]{scrartcl}

\usepackage{amsmath,amssymb}
\usepackage{iftex}
\ifPDFTeX
  \usepackage[T1]{fontenc}
  \usepackage[utf8]{inputenc}
  \usepackage{textcomp} % provide euro and other symbols
\else % if luatex or xetex
  \usepackage{unicode-math}
  \defaultfontfeatures{Scale=MatchLowercase}
  \defaultfontfeatures[\rmfamily]{Ligatures=TeX,Scale=1}
\fi
\usepackage{lmodern}
\ifPDFTeX\else  
    % xetex/luatex font selection
\fi
% Use upquote if available, for straight quotes in verbatim environments
\IfFileExists{upquote.sty}{\usepackage{upquote}}{}
\IfFileExists{microtype.sty}{% use microtype if available
  \usepackage[]{microtype}
  \UseMicrotypeSet[protrusion]{basicmath} % disable protrusion for tt fonts
}{}
\makeatletter
\@ifundefined{KOMAClassName}{% if non-KOMA class
  \IfFileExists{parskip.sty}{%
    \usepackage{parskip}
  }{% else
    \setlength{\parindent}{0pt}
    \setlength{\parskip}{6pt plus 2pt minus 1pt}}
}{% if KOMA class
  \KOMAoptions{parskip=half}}
\makeatother
\usepackage{xcolor}
\setlength{\emergencystretch}{3em} % prevent overfull lines
\setcounter{secnumdepth}{5}
% Make \paragraph and \subparagraph free-standing
\makeatletter
\ifx\paragraph\undefined\else
  \let\oldparagraph\paragraph
  \renewcommand{\paragraph}{
    \@ifstar
      \xxxParagraphStar
      \xxxParagraphNoStar
  }
  \newcommand{\xxxParagraphStar}[1]{\oldparagraph*{#1}\mbox{}}
  \newcommand{\xxxParagraphNoStar}[1]{\oldparagraph{#1}\mbox{}}
\fi
\ifx\subparagraph\undefined\else
  \let\oldsubparagraph\subparagraph
  \renewcommand{\subparagraph}{
    \@ifstar
      \xxxSubParagraphStar
      \xxxSubParagraphNoStar
  }
  \newcommand{\xxxSubParagraphStar}[1]{\oldsubparagraph*{#1}\mbox{}}
  \newcommand{\xxxSubParagraphNoStar}[1]{\oldsubparagraph{#1}\mbox{}}
\fi
\makeatother

\usepackage{color}
\usepackage{fancyvrb}
\newcommand{\VerbBar}{|}
\newcommand{\VERB}{\Verb[commandchars=\\\{\}]}
\DefineVerbatimEnvironment{Highlighting}{Verbatim}{commandchars=\\\{\}}
% Add ',fontsize=\small' for more characters per line
\usepackage{framed}
\definecolor{shadecolor}{RGB}{241,243,245}
\newenvironment{Shaded}{\begin{snugshade}}{\end{snugshade}}
\newcommand{\AlertTok}[1]{\textcolor[rgb]{0.68,0.00,0.00}{#1}}
\newcommand{\AnnotationTok}[1]{\textcolor[rgb]{0.37,0.37,0.37}{#1}}
\newcommand{\AttributeTok}[1]{\textcolor[rgb]{0.40,0.45,0.13}{#1}}
\newcommand{\BaseNTok}[1]{\textcolor[rgb]{0.68,0.00,0.00}{#1}}
\newcommand{\BuiltInTok}[1]{\textcolor[rgb]{0.00,0.23,0.31}{#1}}
\newcommand{\CharTok}[1]{\textcolor[rgb]{0.13,0.47,0.30}{#1}}
\newcommand{\CommentTok}[1]{\textcolor[rgb]{0.37,0.37,0.37}{#1}}
\newcommand{\CommentVarTok}[1]{\textcolor[rgb]{0.37,0.37,0.37}{\textit{#1}}}
\newcommand{\ConstantTok}[1]{\textcolor[rgb]{0.56,0.35,0.01}{#1}}
\newcommand{\ControlFlowTok}[1]{\textcolor[rgb]{0.00,0.23,0.31}{\textbf{#1}}}
\newcommand{\DataTypeTok}[1]{\textcolor[rgb]{0.68,0.00,0.00}{#1}}
\newcommand{\DecValTok}[1]{\textcolor[rgb]{0.68,0.00,0.00}{#1}}
\newcommand{\DocumentationTok}[1]{\textcolor[rgb]{0.37,0.37,0.37}{\textit{#1}}}
\newcommand{\ErrorTok}[1]{\textcolor[rgb]{0.68,0.00,0.00}{#1}}
\newcommand{\ExtensionTok}[1]{\textcolor[rgb]{0.00,0.23,0.31}{#1}}
\newcommand{\FloatTok}[1]{\textcolor[rgb]{0.68,0.00,0.00}{#1}}
\newcommand{\FunctionTok}[1]{\textcolor[rgb]{0.28,0.35,0.67}{#1}}
\newcommand{\ImportTok}[1]{\textcolor[rgb]{0.00,0.46,0.62}{#1}}
\newcommand{\InformationTok}[1]{\textcolor[rgb]{0.37,0.37,0.37}{#1}}
\newcommand{\KeywordTok}[1]{\textcolor[rgb]{0.00,0.23,0.31}{\textbf{#1}}}
\newcommand{\NormalTok}[1]{\textcolor[rgb]{0.00,0.23,0.31}{#1}}
\newcommand{\OperatorTok}[1]{\textcolor[rgb]{0.37,0.37,0.37}{#1}}
\newcommand{\OtherTok}[1]{\textcolor[rgb]{0.00,0.23,0.31}{#1}}
\newcommand{\PreprocessorTok}[1]{\textcolor[rgb]{0.68,0.00,0.00}{#1}}
\newcommand{\RegionMarkerTok}[1]{\textcolor[rgb]{0.00,0.23,0.31}{#1}}
\newcommand{\SpecialCharTok}[1]{\textcolor[rgb]{0.37,0.37,0.37}{#1}}
\newcommand{\SpecialStringTok}[1]{\textcolor[rgb]{0.13,0.47,0.30}{#1}}
\newcommand{\StringTok}[1]{\textcolor[rgb]{0.13,0.47,0.30}{#1}}
\newcommand{\VariableTok}[1]{\textcolor[rgb]{0.07,0.07,0.07}{#1}}
\newcommand{\VerbatimStringTok}[1]{\textcolor[rgb]{0.13,0.47,0.30}{#1}}
\newcommand{\WarningTok}[1]{\textcolor[rgb]{0.37,0.37,0.37}{\textit{#1}}}

\providecommand{\tightlist}{%
  \setlength{\itemsep}{0pt}\setlength{\parskip}{0pt}}\usepackage{longtable,booktabs,array}
\usepackage{calc} % for calculating minipage widths
% Correct order of tables after \paragraph or \subparagraph
\usepackage{etoolbox}
\makeatletter
\patchcmd\longtable{\par}{\if@noskipsec\mbox{}\fi\par}{}{}
\makeatother
% Allow footnotes in longtable head/foot
\IfFileExists{footnotehyper.sty}{\usepackage{footnotehyper}}{\usepackage{footnote}}
\makesavenoteenv{longtable}
\usepackage{graphicx}
\makeatletter
\newsavebox\pandoc@box
\newcommand*\pandocbounded[1]{% scales image to fit in text height/width
  \sbox\pandoc@box{#1}%
  \Gscale@div\@tempa{\textheight}{\dimexpr\ht\pandoc@box+\dp\pandoc@box\relax}%
  \Gscale@div\@tempb{\linewidth}{\wd\pandoc@box}%
  \ifdim\@tempb\p@<\@tempa\p@\let\@tempa\@tempb\fi% select the smaller of both
  \ifdim\@tempa\p@<\p@\scalebox{\@tempa}{\usebox\pandoc@box}%
  \else\usebox{\pandoc@box}%
  \fi%
}
% Set default figure placement to htbp
\def\fps@figure{htbp}
\makeatother

\KOMAoption{captions}{tableheading}
\makeatletter
\@ifpackageloaded{tcolorbox}{}{\usepackage[skins,breakable]{tcolorbox}}
\@ifpackageloaded{fontawesome5}{}{\usepackage{fontawesome5}}
\definecolor{quarto-callout-color}{HTML}{909090}
\definecolor{quarto-callout-note-color}{HTML}{0758E5}
\definecolor{quarto-callout-important-color}{HTML}{CC1914}
\definecolor{quarto-callout-warning-color}{HTML}{EB9113}
\definecolor{quarto-callout-tip-color}{HTML}{00A047}
\definecolor{quarto-callout-caution-color}{HTML}{FC5300}
\definecolor{quarto-callout-color-frame}{HTML}{acacac}
\definecolor{quarto-callout-note-color-frame}{HTML}{4582ec}
\definecolor{quarto-callout-important-color-frame}{HTML}{d9534f}
\definecolor{quarto-callout-warning-color-frame}{HTML}{f0ad4e}
\definecolor{quarto-callout-tip-color-frame}{HTML}{02b875}
\definecolor{quarto-callout-caution-color-frame}{HTML}{fd7e14}
\makeatother
\makeatletter
\@ifpackageloaded{caption}{}{\usepackage{caption}}
\AtBeginDocument{%
\ifdefined\contentsname
  \renewcommand*\contentsname{Table of contents}
\else
  \newcommand\contentsname{Table of contents}
\fi
\ifdefined\listfigurename
  \renewcommand*\listfigurename{List of Figures}
\else
  \newcommand\listfigurename{List of Figures}
\fi
\ifdefined\listtablename
  \renewcommand*\listtablename{List of Tables}
\else
  \newcommand\listtablename{List of Tables}
\fi
\ifdefined\figurename
  \renewcommand*\figurename{Figure}
\else
  \newcommand\figurename{Figure}
\fi
\ifdefined\tablename
  \renewcommand*\tablename{Table}
\else
  \newcommand\tablename{Table}
\fi
}
\@ifpackageloaded{float}{}{\usepackage{float}}
\floatstyle{ruled}
\@ifundefined{c@chapter}{\newfloat{codelisting}{h}{lop}}{\newfloat{codelisting}{h}{lop}[chapter]}
\floatname{codelisting}{Listing}
\newcommand*\listoflistings{\listof{codelisting}{List of Listings}}
\makeatother
\makeatletter
\makeatother
\makeatletter
\@ifpackageloaded{caption}{}{\usepackage{caption}}
\@ifpackageloaded{subcaption}{}{\usepackage{subcaption}}
\makeatother

\usepackage{bookmark}

\IfFileExists{xurl.sty}{\usepackage{xurl}}{} % add URL line breaks if available
\urlstyle{same} % disable monospaced font for URLs
\hypersetup{
  pdftitle={Assignment 03},
  pdfauthor={Bhargavi Manyala},
  colorlinks=true,
  linkcolor={blue},
  filecolor={Maroon},
  citecolor={Blue},
  urlcolor={Blue},
  pdfcreator={LaTeX via pandoc}}


\title{Assignment 03}
\author{Bhargavi Manyala}
\date{September 24, 2025}

\begin{document}
\maketitle


\begin{tcolorbox}[enhanced jigsaw, colback=white, leftrule=.75mm, toprule=.15mm, arc=.35mm, breakable, left=2mm, bottomrule=.15mm, colframe=quarto-callout-color-frame, opacityback=0, rightrule=.15mm]

In my code, I am casting salary and experience columns to float and then
using approxQuantile to compute medians for SALARY\_FROM, SALARY\_TO,
and SALARY. Can you explain what this casting step and the median
calculation are doing, and why they are necessary before filling missing
values?

\end{tcolorbox}

\begin{itemize}
\tightlist
\item
  Great question! Let me break down both the casting and median
  calculation steps:
\end{itemize}

\begin{enumerate}
\def\labelenumi{\arabic{enumi}.}
\tightlist
\item
  Casting to Float
\end{enumerate}

\begin{Shaded}
\begin{Highlighting}[]
\NormalTok{df }\OperatorTok{=}\NormalTok{ df.withColumn(}\StringTok{"SALARY\_FROM"}\NormalTok{, col(}\StringTok{"SALARY\_FROM"}\NormalTok{).cast(}\StringTok{"float"}\NormalTok{))}
\NormalTok{df }\OperatorTok{=}\NormalTok{ df.withColumn(}\StringTok{"SALARY\_TO"}\NormalTok{, col(}\StringTok{"SALARY\_TO"}\NormalTok{).cast(}\StringTok{"float"}\NormalTok{))}
\NormalTok{df }\OperatorTok{=}\NormalTok{ df.withColumn(}\StringTok{"SALARY"}\NormalTok{, col(}\StringTok{"SALARY"}\NormalTok{).cast(}\StringTok{"float"}\NormalTok{))}

\NormalTok{salary\_from\_median }\OperatorTok{=}\NormalTok{ df.approxQuantile(}\StringTok{"SALARY\_FROM"}\NormalTok{, [}\FloatTok{0.5}\NormalTok{], }\FloatTok{0.01}\NormalTok{)[}\DecValTok{0}\NormalTok{]}
\NormalTok{salary\_to\_median }\OperatorTok{=}\NormalTok{ df.approxQuantile(}\StringTok{"SALARY\_TO"}\NormalTok{, [}\FloatTok{0.5}\NormalTok{], }\FloatTok{0.01}\NormalTok{)[}\DecValTok{0}\NormalTok{]}
\NormalTok{salary\_median }\OperatorTok{=}\NormalTok{ df.approxQuantile(}\StringTok{"SALARY"}\NormalTok{, [}\FloatTok{0.5}\NormalTok{], }\FloatTok{0.01}\NormalTok{)[}\DecValTok{0}\NormalTok{]}
\end{Highlighting}
\end{Shaded}

\begin{itemize}
\item
  Why casting is necessary:
\item
  Original data type: These columns are likely stored as strings in the
  raw data
\item
  Mathematical operations: You can't compute medians, averages, or other
  statistics on string data
\item
  Null handling: String columns might have values like ``N/* A'',
  ``null'', or empty strings that need to become actual null values
\item
  Performance: Numeric operations on float columns are much faster than
  string operations
\end{itemize}

\begin{enumerate}
\def\labelenumi{\arabic{enumi}.}
\setcounter{enumi}{1}
\tightlist
\item
  Median Calculation with approxQuantile
\end{enumerate}

\begin{itemize}
\item
  What approxQuantile does:
\item
  {[}0.5{]} = 50th percentile (median)
\item
  0.01 = 1\% approximation error (faster than exact calculation)
\item
  Returns the middle value when data is sorted
\item
  Why median for missing value imputation:
\item
  Robust to outliers: Unlike mean, median isn't skewed by extremely
  high/low salaries
\item
  Representative: Better represents ``typical'' salary in the dataset
\item
  Common practice: Standard approach in data preprocessing for salary
  data
\end{itemize}

\begin{tcolorbox}[enhanced jigsaw, colback=white, leftrule=.75mm, toprule=.15mm, arc=.35mm, breakable, left=2mm, bottomrule=.15mm, colframe=quarto-callout-color-frame, opacityback=0, rightrule=.15mm]

In my dataset, the column EDUCATION\_LEVELS\_NAME contains unwanted
characters like \n and \r (for example: {[}\n ``Bachelor's
degree''\n{]}). How can I remove these characters so that the column
values look clean (e.g., ``Bachelor's degree'') using PySpark or Pandas?

\end{tcolorbox}

\begin{itemize}
\tightlist
\item
  PySpark Solution
\end{itemize}

\begin{Shaded}
\begin{Highlighting}[]
\ImportTok{from}\NormalTok{ pyspark.sql.functions }\ImportTok{import}\NormalTok{ col, regexp\_replace, trim}

\CommentTok{\# Method 1: Chain multiple regexp\_replace calls}
\NormalTok{df }\OperatorTok{=}\NormalTok{ df.withColumn(}\StringTok{"EDUCATION\_LEVELS\_NAME"}\NormalTok{, }
\NormalTok{    regexp\_replace(}
\NormalTok{        regexp\_replace(}
\NormalTok{            regexp\_replace(col(}\StringTok{"EDUCATION\_LEVELS\_NAME"}\NormalTok{), }\VerbatimStringTok{r"[\textbackslash{}[\textbackslash{}]]"}\NormalTok{, }\StringTok{""}\NormalTok{),  }\CommentTok{\# Remove brackets}
            \VerbatimStringTok{r"[\textbackslash{}n\textbackslash{}r]"}\NormalTok{, }\StringTok{""}  \CommentTok{\# Remove newlines and carriage returns}
\NormalTok{        ), }
        \VerbatimStringTok{r\textquotesingle{}\^{}\textbackslash{}s*"|"\textbackslash{}s*$\textquotesingle{}}\NormalTok{, }\StringTok{""}  \CommentTok{\# Remove quotes at start/end}
\NormalTok{    )}
\NormalTok{)}

\CommentTok{\# Method 2: Single regexp\_replace (more efficient)}
\NormalTok{df }\OperatorTok{=}\NormalTok{ df.withColumn(}\StringTok{"EDUCATION\_LEVELS\_NAME"}\NormalTok{,}
\NormalTok{    trim(regexp\_replace(col(}\StringTok{"EDUCATION\_LEVELS\_NAME"}\NormalTok{), }\VerbatimStringTok{r\textquotesingle{}[\textbackslash{}[\textbackslash{}]\textbackslash{}n\textbackslash{}r"\textbackslash{}s]+\textquotesingle{}}\NormalTok{, }\StringTok{" "}\NormalTok{))}
\NormalTok{)}
\end{Highlighting}
\end{Shaded}

\begin{itemize}
\tightlist
\item
  Pandas Solution
\end{itemize}

\begin{Shaded}
\begin{Highlighting}[]
\ImportTok{import}\NormalTok{ re}

\CommentTok{\# Method 1: Using str.replace with regex}
\NormalTok{df[}\StringTok{\textquotesingle{}EDUCATION\_LEVELS\_NAME\textquotesingle{}}\NormalTok{] }\OperatorTok{=}\NormalTok{ df[}\StringTok{\textquotesingle{}EDUCATION\_LEVELS\_NAME\textquotesingle{}}\NormalTok{].}\BuiltInTok{str}\NormalTok{.replace(}\VerbatimStringTok{r\textquotesingle{}[\textbackslash{}[\textbackslash{}]\textbackslash{}n\textbackslash{}r"]\textquotesingle{}}\NormalTok{, }\StringTok{\textquotesingle{}\textquotesingle{}}\NormalTok{, regex}\OperatorTok{=}\VariableTok{True}\NormalTok{).}\BuiltInTok{str}\NormalTok{.strip()}

\CommentTok{\# Method 2: More comprehensive cleaning}
\KeywordTok{def}\NormalTok{ clean\_education(text):}
    \ControlFlowTok{if}\NormalTok{ pd.isna(text):}
        \ControlFlowTok{return}\NormalTok{ text}
    \CommentTok{\# Remove brackets, newlines, quotes, and extra spaces}
\NormalTok{    cleaned }\OperatorTok{=}\NormalTok{ re.sub(}\VerbatimStringTok{r\textquotesingle{}[\textbackslash{}[\textbackslash{}]\textbackslash{}n\textbackslash{}r"]\textquotesingle{}}\NormalTok{, }\StringTok{\textquotesingle{}\textquotesingle{}}\NormalTok{, }\BuiltInTok{str}\NormalTok{(text))}
    \ControlFlowTok{return}\NormalTok{ cleaned.strip()}

\NormalTok{df[}\StringTok{\textquotesingle{}EDUCATION\_LEVELS\_NAME\textquotesingle{}}\NormalTok{] }\OperatorTok{=}\NormalTok{ df[}\StringTok{\textquotesingle{}EDUCATION\_LEVELS\_NAME\textquotesingle{}}\NormalTok{].}\BuiltInTok{apply}\NormalTok{(clean\_education)}
\end{Highlighting}
\end{Shaded}





\end{document}
